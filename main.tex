\documentclass[12pt, letterpaper]{article}
\usepackage[utf8]{inputenc}
 
\title{Tera-Edge Literature Review}
\author{Ajinkya K. Mulay}
\date{Dec. 2018}

\begin{document}
	\begin{titlepage}
		\maketitle
	\end{titlepage}
	
	\section{Present and Future of Terahertz Communications - Ho-Jin Song and Tadao Nagatsuma}
	\textbf{Research Definition:} 
	\par
	To examine the current progress of THz-wave technologies related to communications applications and discuss issues that need to be considered for the future of THz communications. 
	\par
	In recent times, users are consuming more packets via mobile devices rather than through stationary PCs. With the improvements in wireless communications, it is predicted that the speed of future wireless networks will be 10 Gb/s or higher. The improvements in data capacity have been brought about by increasing the spectral efficiency by means of advanced modulation schemes and signal processing techniques. But, due to the fundamental limit caused due to narrow bandwidth the rate of 10 Gb/s is hard to achieve. To increase the spectral resources, we have options of Ultra WideBand (UWB), 60-80 GHz radio, Free-Space Optical (FSO) communications, IrDA along with TeraHertz (THz).
	\par 
	With the advent of femotsecond lasers and photoconductive antennas during the 1980s, it has been possible to use THz for applications in bio- and medical science, pharmacology, security, compound semiconductor devices operating at 1 THz, Si-CMOS technology at frequencies over 100 GHz. These advances make the THz attractive for improving wireless communications, especially the unallocated 275 - 3000 GHz band. Federici and Moeller and Kleiene-Ostmann and Nagatsuma discuss the potential and feasibility of using THz waves in the range 100 GHz - 10 THz for future wireless communications. In this article the authors review the recent progress in fundamental THz technologies along with the issues and barriers for deploying pratical systems in the future. The authors further discuss the feasibility of THz band for wireless communications using a simple link budget calculation and discuss the future THz possibilities along with reviewing the current THz technologies in terms of channel modelling, device technologies for the front end and experimental demonstrations of data transmission at THz frequencies. Finally the authors provide several issues that need to be addressed in the future of THz communications. 
	\subsection{Methodology}
	\subsection{Results}
	


\end{document}