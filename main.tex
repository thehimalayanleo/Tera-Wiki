\documentclass[12pt, letterpaper]{article}
\usepackage[utf8]{inputenc}
\usepackage{amsmath} 
\title{Tera-Edge Literature Review}
\author{Ajinkya K. Mulay}
\date{Dec. 2018}

\begin{document}
	\begin{titlepage}
		\maketitle
	\end{titlepage}
	
	\section{Present and Future of Terahertz Communications Review}
\par
The purpose of this paper is to provide a feasibility study of TeraHertz (THz) communications as a front-runner for complimenting the current wireless communication technologies. It is primarily motivated due to the sudden burst in mobile devices along with high-quality media streaming. The work also states that even though THz has huge bandwidth availability until recently there has been a lack of devices working in the THz frequency domain. The authors have provided a thorough review of the current high-frequency devices to address these concerns. Finally, the work lists the potential challenges in implementing THz systems.
\par
The authors consider the effects of atmospheric attenuation at high-frequency communications and observe that any frequency above 1 THz will lead to significant losses.  Thus, the work only considers indoor applications. For calculating the data capacity of a THz communication system, the authors employ a simple link budget calculation using the Friis formula. The authors conclude that with parameters achievable using current devices, the antenna will be very small in size and thus will require Line-Of-Sight (LOS) communication for efficiency. 
\par
Indoor applications such as home wireless networks for digital video, "triple-stack nano-cellular architecture" mobile devices and battlefield secure wireless systems have been proposed in this work. To demonstrate the immediate future of THz, the authors review the current promising literature based on THz channel modeling and device technologies. Since astronomical applications also operate in the THz domain, the paper discusses its impact on wireless THz devices by providing the safe distances outside which the interference between these technologies will negligible. As mentioned above due to the LOS operation of THz, its necessary to look into beam steering technology and the paper predicts that technologies used for microwave or infrared (\emph{i.e.,} Lens/Mirror approach made by MEMS or meta-material) should be compatible. Another possible challenge is the packaging used for the THz devices and due to factors of losses and practicality, it will be necessary to develop better THz integrated circuits. 
\par
The major conclusion of this work is that for THz communications to be feasible we require a LOS, indoor application which is at a safe distance from astronomical setups to reach a data rate of more than 10 Gb/s. The prediction is that with a better understanding of THz channel characteristics, a data rate greater than 100 Gb/s will be achievable. Finally, we need to verify that the current beam steering technologies work for THz as well and need to develop more efficient and practical device packaging. 
\par
This work, however, does not address the practical scenarios of non-LOS or outdoor applications of THz which are much more lucrative for the future of wireless communications. It does provide a good starting point in understanding the current possibilities in THz technologies and possible issues that need to be addressed immedietely.
\end{document}