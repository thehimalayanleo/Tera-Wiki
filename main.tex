\documentclass[12pt, letterpaper]{article}
\usepackage[utf8]{inputenc}
\usepackage{amsmath} 
\title{Tera-Edge Literature Review}
\author{Ajinkya K. Mulay}
\date{Dec. 2018}

\begin{document}
	\begin{titlepage}
		\maketitle
	\end{titlepage}
	
	\section{Present and Future of Terahertz Communications Review \cite{song2011present}} 
\par
The purpose of this paper is to provide a feasibility study of TeraHertz (THz) communications as a front-runner for complimenting the current wireless communication technologies. It is primarily motivated due to the sudden burst in mobile devices along with high-quality media streaming. The work also states that even though THz has huge bandwidth availability until recently there has been a lack of devices working in the THz frequency domain. The authors have provided a thorough review of the current high-frequency devices to address these concerns. Finally, the work lists the potential challenges in implementing THz systems.
\par
The authors consider the effects of atmospheric attenuation at high-frequency communications and observe that any frequency above 1 THz will lead to significant losses.  Thus, the work only considers indoor applications. For calculating the data capacity of a THz communication system, the authors employ a simple link budget calculation using the Friis formula. The authors conclude that with parameters achievable using current devices, the antenna will be very small in size and thus will require Line-Of-Sight (LOS) communication for efficiency. 
\par
Indoor applications such as home wireless networks for digital video, "triple-stack nano-cellular architecture" mobile devices and battlefield secure wireless systems have been proposed in this work. To demonstrate the immediate future of THz, the authors review the current promising literature based on THz channel modeling and device technologies. Since astronomical applications also operate in the THz domain, the paper discusses its impact on wireless THz devices by providing the safe distances outside which the interference between these technologies will negligible. As mentioned above due to the LOS operation of THz, its necessary to look into beam steering technology and the paper predicts that technologies used for microwave or infrared (\emph{i.e.,} Lens/Mirror approach made by MEMS or meta-material) should be compatible. Another possible challenge is the packaging used for the THz devices and due to factors of losses and practicality, it will be necessary to develop better THz integrated circuits. 
\par
The major conclusion of this work is that for THz communications to be feasible we require a LOS, indoor application which is at a safe distance from astronomical setups to reach a data rate of more than 10 Gb/s. The prediction is that with a better understanding of THz channel characteristics, a data rate greater than 100 Gb/s will be achievable. Finally, we need to verify that the current beam steering technologies work for THz as well and need to develop more efficient and practical device packaging. 
\par
This work, however, does not address the practical scenarios of non-LOS or outdoor applications of THz which are much more lucrative for the future of wireless communications. It does provide a good starting point in understanding the current possibilities in THz technologies and possible issues that need to be addressed immedietely.
\section{Terhertz Communication for Vehicular Networks Reviews \cite{mumtaz2017terahertz}} 
\par
The primary goal of this paper is to lay down the role of TeraHertz (THz) communication in helping reach the goals for Beyond the Fifth Generation (B5G) mobile networks in terms of high data rates per device and per area, connecting a large number of devices and improving reliability for critical applications. The unavailability of very wide bands in frequencies below 90 GHz motivates the need for THz bands. Even though THz has some crucial issues in propagation and atmospheric absorption, it clearly stands out due to its high bandwidth as well as the ability to fit a large number of antennas (due to its smaller wavelength) thus enabling the creation of advanced adaptive arrays. THz will allow instantaneous data transfer of the magnitude of 1 Terabit in 1 second (in an ideal scenario) and also reduce the Doppler effect's impact thus making them beneficial for vehicular networks.
\par
The authors point out that due to a high path loss in THz bands an upgraded transceiver design capable of high power, high sensitivity, and a low noise figure is required. THz has the ability to host a large array of antenna elements, but there is a lack of a work which completely describes the potential new THz MIMO transceiver. This array of antennas leads to multiple other challenges of developing feeding and control networks and also analyzing the coupling effects between neighboring antennas. The huge quantity of antennas means that the number of terminals might be lesser than the antennas, thus enabling a simple linear pre-coding in the forward link. Thus, the authors believe that THz will enable information theory to provide a better trade-off between spectral and energy efficiency. However, in THz, the number of terminals supported per device is lower due to a smaller range and the channel coherence is smaller too (due to larger Doppler spread). 
\par
For THz, an approach involving orthogonal channels is proposed as part of its medium access mechanism which minimizes the transceiver complexity while maximizing the channel capacity. However, the lack of energy efficiency (especially due to MIMO) in Orthogonal Frequency-Division Multiplexing (OFDM) negates this advantage. There is also interesting work ongoing using Single-Carrier Modulation (SCM) for providing near-optimal sum rate performance in MIMO systems for low-transmit-power-to-receiver-noise-power ratios. However, this work does not entirely align with the characteristics of THz channel and thus requires further understanding. In the case of THz, especially in urban scenarios and due to high path loss and directional antenna beams, better 3D models are required. The smaller coherence interval possesses a challenge for obtaining error-free Channel State Information (CSI).  A possible solution is using near Line-Of-Sight (LOS) propagation coupled with the high path loss of THz bands. A LOS channel will also help in channel estimation via Direction-Of-Arrival (DOA) estimation. However, the trade-offs between DOA estimation and the added DOA complexity and array calibration needs to be studied.
\par
Due to the smaller coherence time, the MAC layer decisions need to be taken more frequently. This fact coupled with the other specialties of THz - huge number of antennas, special propagation features and hardware required - entails us to create a separate MAC layer for THz bands. For vehicular networks, mobility is another factor to be considered. THz has a number of interference mitigation techniques inbuilt like increased path loss allowing for high frequency reuse, shadowing LOS effects reducing leakage to neighboring cells, large spectrum availability and beamforming along with high spatial selectivity reduces the probability of interference. However, in THz, the interference possibilities arise in high gain beam steering (especially for full frequency reuse cases) and physical blockage and these might be solved via the large degrees of freedom available in massive MIMO (\emph{i.e., } using subspace-based and interference alignment techniques).
\par
The work states the importance of Random Matrix Theory (RMT) for mathematical analysis in THz since its used significantly in MIMO communications. THz could also potentially replace wired back-hauls especially in locations where wired connections are costly and tough to maintain. Using beam steering they could potentially provide a backhaul to multiple stations.  
\section{A Review on Terahertz Communications Research \cite{kleine2011review}} 
\par
This paper aims to shed clarity on the physical layer upgrades required for developing a TeraHertz (THz) communication system capable of transmitting with data rates far higher than 10 Gb/s. The motivation for developing these systems is the vast increase in high-quality video traffic.  The infrared frequency range which lies above the THz has however been ruled out due to its limitations of the low sensitivity of incoherent receivers, high diffuse reflection losses, high ambient light noise, and low power options because of eye-safety issues. Immediate applications of THz communication involves high-quality video streaming, Transfer Jets or Giga-IR for short-range, high-speed data bursts and instantaneous data transfer between personal mobiles with servers given the increase in disk speeds due to SSDs. 
\par
The work further discusses the general hurdles for THz links which include huge losses and quasi-optical propagation path. Also, the $H_2O$ molecules possess an extra attenuation factor for these THz waves. The authors thus conclude that such a high loss path reduces the probability of a long-range outdoor THz network while also making the high-gain antennas necessary for successful transmissions. In an outdoor high-gain antenna scenario, a stationary outdoor link using parabolic antennas might be feasible. Similarly in a wireless interconnection of different devices like laptops and kiosks horn antennas with rough steering are possible too. However, for an indoor pico-cellular link there are further hurdles of possessing an electronically steerable phased array antenna for automatically connecting the communicating devices along while having the ability to communicate over non-line-of-sight (LOS) links by using reflections from the surroundings (due to obstacles like humans in the middle). The authors stress the importance of smart antennas as well as MIMO in improving the reliability, coverage, and capacity of such a system.  
\par
The paper discusses three approaches to develop potential THz transmitters - an all electronics approach, a photonic technique and finally using lasers such as quantum cascade lasers. Similarly, it provides options for THz receivers involving direct detection using Schottky barrier diodes and heterodyne detection using a Schottky barrier diode mixer with a local oscillator (LO). The components required for these devices include the opta/electrical (O/E) converter, especially for photonics-based transmitters. A high-output power photodiode called a uni-traveling-carrier photodiode (UTC-PD) is useful in integration with a plasma antenna in place of the conventional one due to its higher output power. However, since the power decreases with $f^{-4}$, an array of antennas was used to obtain a higher power. Issues of thermal management in UTC-PDs are solved using Si substrate by wafer bonding as well as a metal-metal bonding.
\par
Due to issues of unprecedented obstacles in indoor THz communications, dielectric plastic mirrors are proposed (for use on the walls) to reflect the THz frequencies without blocking other communication frequencies like GSM/UMTS instead of using thin metal films which might hinder the other frequencies. Further, these mirrors need good reflective over a broad range of incident angles during non-LOS path communications. Just covering a small percentage of the walls or ceilings with such a dielectric material the signal level improves significantly while also reducing the antenna gain required. As discussed before the need for adjustment in indoor scenarios makes horn antennas, open-ended waveguides or parabolic antennas unusable in most cases. As the directivity of the antenna increases, the time and effort needed for adjustment increases as well. Thus, applications allowing for a bit of adjustment can still well use the types of antennas. But, most of the time, electronically steerable antennas will be required to allow for fast automatic reconfiguration or for establishing several transmission paths. The cost-effectiveness and easy fabrication, flexibility, and potential for integration make planar structures great for commercial communication systems.
\par
External THz modulators help in the modulation of carrier waves with data streams, adaptive beamforming, and electronically switchable mirrors or filters. They might be constructed using semiconductor heterostructures or liquid crystals with the primary purpose of modulating the magnitude and phase of reflected and/or transmitted signals. For actually improving the data rates beyond 10 Gb/s, along with the availability of more massive bandwidth improvement in spectral efficiency is necessary. Spectral efficiency is improved using either complex modulation schemes (better SINR) or MIMO (multi-path propagation. However, actual performance depends on the channel properties like path loss, antenna misalignment and interference caused due to reflection and/or scattering. The channel impulse response is calculated using the channel properties, which can further be used to determine the pulse delay spread (temporal spreading of a sharp pulse in the channel) which can help in assessing the data rate. 
\par
To obtain the reliable channel simulation results, ray-tracing algorithms can be used (already used in mm-waves) especially in indoor communication systems involving multiple propagation paths. These have been used in real-life scenarios to determine the potential required antenna gains and achievable data rates along with assessing the benefits of the dielectric mirrors. Multiple systems to accurately determine the channel characteristics of a THz communication system have been discussed in this work to demonstrate the feasibility of THz.
\par
Due to the attenuation of THz communications, three valleys of minimum attenuation at 75-100, 110-150 and 220-270 GHz are identified with 120 GHz band centered at 125 GHz being the first choice. To improve the data rate to more than 20 Gb/s bands in the range higher than 300 GHz have been identified. A photonics RF transmitter has the advantage of being compact and light-weight, being expandable to a multi-band system while working on one of the minimum attenuation bands. The authors also describe devices for raising the data rate to around 20 Gb/s using frequencies higher than 300 GHz.
\par
The authors expect that the need for a 10 Gb/s link is immediate due to the fast rise in HDTV signals and foresee the need for 20, 40 and 100 Gb/s to follow suit in the future with the growth of Super Hi-Vision (SHV) and Ultra-high definition (UHD) TV data having resolutions 16 times that of HDTV. Further, the need for quick data transfer in the near field also prompts the demand for potentially super-fast technologies like THz. ASK, and PSK has shown good compatibility with THz frequencies and the unallocated frequencies above 275 GHz are especially interesting. Development in device technology has predicted that the cut-off frequency of Si-CMS will exceed 500 GHz soon. Due to high frequencies, the antenna size reduces significantly to a sub-millimeter range making the antenna array extremely compact and cheap to fit into devices. Due to atmospheric attenuation, the 500 GHz frequency seems like the limit for the outdoor applications, but not for the indoor applications. The clash of passive services like radio astronomy in the 275-3000 GHz range has prompted the study of interference effects between THz and radio astronomy. 

\section{Energy and spectrum-aware MAC protocol for perpetual wireless nanosensor networks in the Terahertz Band \cite{wang2013energy}}  
\par
The primary aim of this paper is developing MAC protocols for nano-devices. The two main challenges in doing so is the need to use the designated frequency band - THz - for nano-devices along with the extreme energy limitations of the devices. The interest in the devices is motivated by the fact that these devices can provide very specific functionality and the ability to do this at a nano-scale. One such application is nano-sensing and the subsequent development of Wireless NanoSensor Networks (WNSN) which has important use-cases in bio-medical (health monitoring and drug delivery systems), defense systems (anti-biological and anti-chemical defenses) and environmental research (plagues and air pollution control). The miniature size of nano-devices make only the THz band a feasible band for communication mainly due to the antenna size. While THz has a very large bandwidth which allows for the possibility of simple modulation and medium access schemes, it suffers from high propagation loss limiting its communication range. Further, the energy limitation and the impracticality of switching batteries regularly have motivated the development of nano-scale energy harvesting techniques. With the nano-scale energy generators, the energy of nano-devices fluctuates in both the positive and the negative side and thus jointly optimizing the energy harvesting and consumption is extremely beneficial.
\par
Overall the three reasons to develop special MAC layer techniques for these devices are the limited processing capacity leading to the need for ultra-low complexity algorithms, handling the special properties of the THz bands, specifically the synchronization issues along with the high propagation loss and finally the temporary energy fluctuations caused due to the behavior of the harvesting systems. The authors propose energy and spectrum aware MAC algorithm which utilizes the hierarchical structure of the nano-devices network over a pulse-based physical layer technique (a more practical approach) and also over an idealistic bandwidth-adaptive channel optimized PHY layer. The authors also propose a metric called CTR (Critical Packet Transmission Ratio) which defines the ratio between the transmission time and the energy harvesting time below which more energy is harvested than is utilized. Using this metric both the previously mentioned algorithms are proposed. Also, after reviewing the existing literature for WNSN the authors conclude that the existing work either does not fully utilize the benefits of THz (or sometimes even negates them) or the special characteristics of energy harvesting systems thus making the need for a new MAC layer paramount.
\par
The network model is developed in a hierarchical manner which can thus be used to shift the computational load from the nano-sensors to the more resourceful nano-controllers. Due to the low range of THz communications, the authors propose a bandwidth-adaptive distance-based communication scheme. Since the energy harvesters have positive and negative fluctuations, the work has developed a jointly optimized scheme for both the energy consumption and harvesting leading to a perpetual energy source and thus eliminating the need for changing batteries. The proposed hierarchical network structure consists of clusters of nano-sensors each headed by a nano-controller with more resources along with the ability to synchronize THz communication. Even though THz has high bandwidth availability, the conventional MAC protocols include heavy signaling thus limiting the achievable throughput. Here the nano-controller takes care of the network synchronization and a dynamic TDMA based scheduling algorithm is proposed. Each nanosensor is assigned a variable-length transmission time slots depending on their data to transmit, the distance from the controller and the channel conditions along with the battery available in the sensor. Other times the nanosensor is sleeping and the energy harvesting takes place during both sleeping and transmission. The work optimizes the way the transmission and the sleeping slots are assigned in order to create a network not requiring a change in batteries. Further, this scheme is also required to provide single-user throughput optimality simultaneously for each sensor. Each frame is divided into three types of sub-frames - downlink (DL), uplink (UL), and Random Access (RA). In the DL subframe, the controller sends broadcast (BC) general information to the sensors and can also send targeted information to specific sensors. It can also be used to send wake up preambles. In the UL sensors send their data to the controller while in the RA, sensors can require the slots to the controller for the next frame or can exchange information in an ad-hoc fashion if the protocol supports it. In this work, the authors only focus on the UL subframe since it affects the power and resource use the most. The need for sending data to the controller is informed in the RA subframe by the sensor and then the controller determines when to send it and sends this information in the DL. The sensor needs to send its ID, amount of data and remaining energy. The controller uses this request along with the channel conditions to determine the way the sensor should send data.
\par
To achieve a fair throughput and optimal channel access to these devices, they devise a joint optimization of energy harvesting and consumption. Using a newly proposed pulse-based physical layer technique, the authors are able to develop a symbol-compression scheduling algorithm which allows the nano-devices to transmit in parallel without collisions. They also present a packet-level timeline scheduling algorithm with which they are able to achieve a balanced single-user throughput along without the need to change the batteries.

\bibliography{bibliography} 
\bibliographystyle{ieeetr}

\end{document}