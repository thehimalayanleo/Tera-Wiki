\documentclass[12pt, letterpaper]{article}
\usepackage[utf8]{inputenc}
\usepackage{amsmath} 
\title{Tera-Edge Literature Review}
\author{Ajinkya K. Mulay}
\date{Dec. 2018}

\begin{document}
	\begin{titlepage}
		\maketitle
	\end{titlepage}
	
	\section{Present and Future of Terahertz Communications Review}
\par
The purpose of this paper is to provide a feasibility study of TeraHertz (THz) communications as a front-runner for complimenting the current wireless communication technologies. It is primarily motivated due to the sudden burst in mobile devices along with high-quality media streaming. The work also states that even though THz has huge bandwidth availability until recently there has been a lack of devices working in the THz frequency domain. The authors have provided a thorough review of the current high-frequency devices to address these concerns. Finally, the work lists the potential challenges in implementing THz systems.
\par
The authors consider the effects of atmospheric attenuation at high-frequency communications and observe that any frequency above 1 THz will lead to significant losses.  Thus, the work only considers indoor applications. For calculating the data capacity of a THz communication system, the authors employ a simple link budget calculation using the Friis formula. The authors conclude that with parameters achievable using current devices, the antenna will be very small in size and thus will require Line-Of-Sight (LOS) communication for efficiency. 
\par
Indoor applications such as home wireless networks for digital video, "triple-stack nano-cellular architecture" mobile devices and battlefield secure wireless systems have been proposed in this work. To demonstrate the immediate future of THz, the authors review the current promising literature based on THz channel modeling and device technologies. Since astronomical applications also operate in the THz domain, the paper discusses its impact on wireless THz devices by providing the safe distances outside which the interference between these technologies will negligible. As mentioned above due to the LOS operation of THz, its necessary to look into beam steering technology and the paper predicts that technologies used for microwave or infrared (\emph{i.e.,} Lens/Mirror approach made by MEMS or meta-material) should be compatible. Another possible challenge is the packaging used for the THz devices and due to factors of losses and practicality, it will be necessary to develop better THz integrated circuits. 
\par
The major conclusion of this work is that for THz communications to be feasible we require a LOS, indoor application which is at a safe distance from astronomical setups to reach a data rate of more than 10 Gb/s. The prediction is that with a better understanding of THz channel characteristics, a data rate greater than 100 Gb/s will be achievable. Finally, we need to verify that the current beam steering technologies work for THz as well and need to develop more efficient and practical device packaging. 
\par
This work, however, does not address the practical scenarios of non-LOS or outdoor applications of THz which are much more lucrative for the future of wireless communications. It does provide a good starting point in understanding the current possibilities in THz technologies and possible issues that need to be addressed immedietely.
\section{Terhertz Communication for Vehicular Networks Reviews}
\par
The primary goal of this paper is to lay down the role of TeraHertz (THz) communication in helping reach the goals for Beyond the Fifth Generation (B5G) mobile networks in terms of high data rates per device and per area, connecting a large number of devices and improving reliability for critical applications. The unavailability of very wide bands in frequencies below 90 GHz motivates the need for THz bands. Even though THz has some crucial issues in propagation and atmospheric absorption, it clearly stands out due to its high bandwidth as well as the ability to fit a large number of antennas (due to its smaller wavelength) thus enabling the creation of advanced adaptive arrays. THz will allow instantaneous data transfer of the magnitude of 1 Terabit in 1 second (in an ideal scenario) and also reduce the Doppler effect's impact thus making them beneficial for vehicular networks.
\par
The authors point out that due to a high path loss in THz bands an upgraded transceiver design capable of high power, high sensitivity, and a low noise figure is required. THz has the ability to host a large array of antenna elements, but there is a lack of a work which completely describes the potential new THz MIMO transceiver. This array of antennas leads to multiple other challenges of developing feeding and control networks and also analyzing the coupling effects between neighboring antennas. The huge quantity of antennas means that the number of terminals might be lesser than the antennas, thus enabling a simple linear pre-coding in the forward link. Thus, the authors believe that THz will enable information theory to provide a better trade-off between spectral and energy efficiency. However, in THz, the number of terminals supported per device is lower due to a smaller range and the channel coherence is smaller too (due to larger Doppler spread). 
\par
For THz, an approach involving orthogonal channels is proposed as part of its medium access mechanism which minimizes the transceiver complexity while maximizing the channel capacity. However, the lack of energy efficiency (especially due to MIMO) in Orthogonal Frequency-Division Multiplexing (OFDM) negates this advantage. There is also interesting work ongoing using Single-Carrier Modulation (SCM) for providing near-optimal sum rate performance in MIMO systems for low-transmit-power-to-receiver-noise-power ratios. However, this work does not entirely align with the characteristics of THz channel and thus requires further understanding. In the case of THz, especially in urban scenarios and due to high path loss and directional antenna beams, better 3D models are required. The smaller coherence interval possesses a challenge for obtaining error-free Channel State Information (CSI).  A possible solution is using near Line-Of-Sight (LOS) propagation coupled with the high path loss of THz bands. A LOS channel will also help in channel estimation via Direction-Of-Arrival (DOA) estimation. However, the trade-offs between DOA estimation and the added DOA complexity and array calibration needs to be studied.
\par
Due to the smaller coherence time, the MAC layer decisions need to be taken more frequently. This fact coupled with the other specialties of THz - huge number of antennas, special propagation features and hardware required - entails us to create a separate MAC layer for THz bands. For vehicular networks, mobility is another factor to be considered. THz has a number of interference mitigation techniques inbuilt like increased path loss allowing for high frequency reuse, shadowing LOS effects reducing leakage to neighboring cells, large spectrum availability and beamforming along with high spatial selectivity reduces the probability of interference. However, in THz, the interference possibilities arise in high gain beam steering (especially for full frequency reuse cases) and physical blockage and these might be solved via the large degrees of freedom available in massive MIMO (\emph{i.e., } using subspace-based and interference alignment techniques).
\par
The work states the importance of Random Matrix Theory (RMT) for mathematical analysis in THz since its used significantly in MIMO communications. THz could also potentially replace wired back-hauls especially in locations where wired connections are costly and tough to maintain. Using beam steering they could potentially provide a backhaul to multiple stations.  

\end{document}