\documentclass[12pt, letterpaper]{article}
\usepackage[utf8]{inputenc}
\usepackage{amsmath} 
\title{Tera-Edge Literature Review}
\author{Ajinkya K. Mulay}
\date{Dec. 2018}

\begin{document}
	\begin{titlepage}
		\maketitle
	\end{titlepage}
	
	\section{Present and Future of Terahertz Communications - Ho-Jin Song and Tadao Nagatsuma}
	\textbf{Research Definition:} 
	\par
	To examine the current progress of THz-wave technologies related to communications applications and discuss issues that need to be considered for the future of THz communications. 
	\par
	In recent times, users are consuming more packets via mobile devices rather than through stationary PCs. With the improvements in wireless communications, it is predicted that the speed of future wireless networks will be 10 Gb/s or higher. The improvements in data capacity have been brought about by increasing the spectral efficiency by means of advanced modulation schemes and signal processing techniques. But, due to the fundamental limit caused due to narrow bandwidth the rate of 10 Gb/s is hard to achieve. To increase the spectral resources, we have options of Ultra WideBand (UWB), 60-80 GHz radio, Free-Space Optical (FSO) communications, IrDA along with TeraHertz (THz).
	\par 
	With the advent of femotsecond lasers and photoconductive antennas during the 1980s, it has been possible to use THz for applications in bio- and medical science, pharmacology, security, compound semiconductor devices operating at 1 THz, Si-CMOS technology at frequencies over 100 GHz. These advances make the THz attractive for improving wireless communications, especially the unallocated 275 - 3000 GHz band. Federici and Moeller and Kleiene-Ostmann and Nagatsuma discuss the potential and feasibility of using THz waves in the range 100 GHz - 10 THz for future wireless communications. In this article the authors review the recent progress in fundamental THz technologies along with the issues and barriers for deploying pratical systems in the future. The authors further discuss the feasibility of THz band for wireless communications using a simple link budget calculation and discuss the future THz possibilities along with reviewing the current THz technologies in terms of channel modelling, device technologies for the front end and experimental demonstrations of data transmission at THz frequencies. Finally the authors provide several issues that need to be addressed in the future of THz communications. 
	\subsection{Methodology}
	\subsubsection{Atomshperic Attenuation}
	Large signal loss due to atomspheric attenuation at frequencies above 100 THz rather than at microwaves. This degrades service converge as well as SINR and influences data capacity. Frequency region above 1 THz is unsuitable for wireless communications because of the absorption lines of $H_2O$ and other gases. There are multiple frequency windows between water absorption lines below 1 THz, but it would be difficult to use these waves as a carrier dye to the losses caused by weather conditions. For indoor applications, this is not a problem though and will have negligible effect.
		\subsubsection{Link Budget for Short Range Applications}
		\begin{align*}
			&P_r = P_t + G_t + G_r + 20  log(\frac{\lambda_c}{4 \pi d}) - \{ \alpha_a (f_c) d\} - L_{ex} \\
			& SNR_{dB} = P_r - (N_0 + 10og(B)) + NF + M)
		\end{align*}, where $P_t$ is the input power, $G_t$ and $G_r$ are the antenna gains of the transmitting and receiving antennas respectively. $D$ and $\lambda$ are the distance and the wavelength. $\alpha_a(f)$ is the atmospheric attenutation at frequency $f$. $L_{ex}$ is an excess loss not included in free-space loss. NF and M are the total noise figue of the receiver and system margin in decibels. $N_0$ is the noise power spectral density. B iis the system noise bandwidth. By the results obtained it can be seen that for microwave frequencies an anetnna of gain 30-dBi will be required which is impractical for indoor applications. However, at THz frequencies the size will not be a problem, since the effective apperture will be around 80 $mm^2$. Since the half-power beam width for the 30-dBi antennna is quite narrow, the transmitter and receiver should be on line-of-sight (LOS) operation for proper operations. To avoid link failure due to an obstacle in the LOS path, walls, ceilings of floors can be used as reflectors so that the link can be kept alive. For a non-LOS operation, the additional loss due to the reflection should be considered in the link budget calculation as well. The current devices are able to hanlde the parameters used in the link budget calculation. However, advanced devices and better modulation schemes will help in achieving 100 Gb/s throughputs as well.
		\subsubsection{Application Scene of THz Communications}
		One crucial scenario for THz is in the WLAN domain of providing digital video on the home wireless networks. Even though the current HDTV format requires around 3-Gb/s dta rate, the required rate for future Super Hi-Vision requires around 24 Gb/s dara rate. There will be certain issues with beam forming especially due to person's moving around. A triple-stack-nanoarchitecture has bee proposed by Britz for mobile devices with Wi-Fi and cellular offering low rate wide-coverage while the THz communications will be offered when they are available. Also, Koch and Federici \textit{et al.} have proposed security systems for use on the battlefield. Due to the highly diretional nature of the THz beams along with the high atmosphetic attenuation, unauthorized persons have to be on the same narrow beam to intercept messages. The data signal can also be spread over a wide spectral span in the THz bands using a long code sequence (around a rate of 100 Gb/s). The high order encryption and the long code would make the system strong against a jamming attack.
	\subsection{Review of current technologies}
	\subsubsection{Channel Modelling}
	Few reports on channel mesaurements or characterization in the THz frequencies. This is due to the low sensetivity of present measurement systems is not high enough in this frequency region. Measurements were carried out on building materials with rough surfaces by a German group involved in the Terahertz Communications Laboratory with the help of Kichoff's scattering theory and Fresnel equations. Similarly work by Priebe \textit{et al.} measured channel properties and compared the results with simulations based on ray-tracing techniques at 300 GHz. This work considered the effect from the building materials work above and the measured and simulated channel impulse reponses showed good similarities. 
	\subsubsection{Device Technologies for the front end}
	Compound semiconductor transistors are considered the strongest candidates for high-frequency applications. InP HEMT devices with a cutoff frequency over 1 THz have been demonstrated for amplifiers and other elementary components at upto around 500 GHz and 10 Gb/s QPSK modulator and demodulator chip sets for a 120-GHz band have been demonstrated. Other non-communication applications include a single chip receiver integrated with a mixer, amplifier and frequency multiplier operating at 220 GHz band, solid-state power amplifiers provide more than 50 mW at 220 GHz and more than 10 mW at 338 GHz. The current technologies support 10 mW at 300 GHz which is ten times the power assumed in the link budget calculation.	
	\subsection{Future of Terahertz}
\end{document}